\documentclass[8pt,oneside, german ,a4paper, fleqn, titlepage]{report}
%{book,memoir,article}
\usepackage[latin1]{inputenc}
\usepackage{amsmath}
\usepackage{amsfonts}
\usepackage{amssymb}
\usepackage{graphicx}
\usepackage{german}
\usepackage{theorem}
\usepackage{mathrsfs} %mathcal , mathscr

\pagestyle{headings}
\parindent0mm

\begin{document}
	
	\title{Everything I need for AI and ML}
	\author{Ruediger Bachmann} 
	\maketitle


\chapter{Texercises}

\section{Text Formatting and Sections}
{\em Hello World} 

Hello World 2

{\bf Hello Mofo}

{\Large \bf Hello again }

\section{Section}
\subsection{Subscection}
\subsubsection{SubSub}
	
\section{The Math Environment}

\subsection{Some Formatting}

$a^2$ \newline
\begin{math}
\delta \Delta : \rightarrow \newline
\mathscr{IIIIIIIIIIII}  \newline
\mathscr IIIII\newline
\mathfrak{UIUIUIUIUI}	\newline
\mathcal{U I U I } \newline
\displaystyle\sum_{i=0}^{\infty}
f(i)\frac{\partial u}{\partial x}
\sqrt[4]{3}
\end{math}  \newline


  
Eine {\bf Inzidenzstruktur} $I$ ist ein Tripel $I=({\cal{P,L,I}})$ mit  
\label{Inzidenz}

\subsection{Equations}
\ldots it is clear that
\begin{equation}
\epsilon > 0.
\label{eq:eps}
\end{equation}

\section{References}
\subsection{Try to refer something}
As mentioned in section \ref{Inzidenz} in page \pageref{Inzidenz} \textellipsis  \newline
From equation~\ref{eq:eps} it follows that \ldots







  
\begin{thebibliography}{99}
\addcontentsline{toc}{chapter}{Literaturverzeichnis}
\bibitem{HopcUllm} John E. Hopcroft and Jeffrey D. Ullman. Einf"uhrung in die Automatentheorie, Formale Sprachen und Komplexit"atstheorie. Addison-Wesley, 1990



\bibitem{BeuE1} A. Beutelspacher. Einf"uhrung in die endliche 
Geometrie I. Blockpl"ane. Bibliographisches Institut, Mannheim --
Wien -- Z"urich, 1982
\bibitem{BeuE2} A. Beutelspacher. Einf"uhrung in die endliche 
Geometrie II. Projektive R"aume. Bibliographisches Institut, Mannheim --
Wien -- Z"urich, 1983
\bibitem{BeuRos} A. Beutelspacher und U. Rosenbaum. Projektive Geometrie. 
Von den Grundlagen bis zu den Anwendungen. Braunschweig -- Wiesbaden, 
Vieweg, 1992
\bibitem{BosBur} R. C. Bose und R. C. Burton. A characterization of flat spaces in a
finite geometry and the uniqueness of the Hamming and the McDonald codes.
J. combinat. Theory 1, 96 -- 104, 1966
\bibitem{Cam1} P. J. Cameron. PROJECTIVE AND POLAR SPACES. QMW Maths Notes 13,
University of London, 1993
\bibitem{Feld} F. D. Feldkamp. Polar geometry, I -- V. Proc. Kon. Ned. Akad. Wet.
A62 512 -- 551; A63, 207 -- 212 (=Indag. Math., 21, 22), 1959
\bibitem{Hirsch1} J. W. P. Hirschfeld. Projective geometries over finite fields.
Oxford University Press, Oxford, 1979
\bibitem{LidNie} R. Lidl und H. Niederreiter. Finite Fields. Addison -- Wesley,
Reading, Mass., 1983
\bibitem{Lorenz2} F. Lorenz. Lineare Algebra II. Bibliographisches Institut,
Mannheim -- Wien -- Z"urich, 1982
\bibitem{Met} K. Metsch. Vorlesung Polarr"aume. Mathematisches Institut der
JLU Gie"sen, SS 1995
\bibitem{Tits} J. Tits. Buildings of Spherical Type and Finite BN -- Pairs.
Lecture Notes in Math. 386, Springer Verlag, Berlin -- Heidelberg -- New York,
1974
\end{thebibliography}
	
\end{document}comment